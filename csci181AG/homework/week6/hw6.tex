\documentclass[12pt]{article}
\usepackage{geometry}
\usepackage{enumerate}
\usepackage{amsmath}
\usepackage{amssymb}
\usepackage{etoolbox}
\usepackage{graphicx}
\usepackage{framed}
\usepackage{tikzsymbols}

\newcommand{\WT}[1]{\begin{framed} \noindent \textbf{What's the Rationale?} #1 \end{framed}}

% Turn solutions and grading guidelines on or off. 
\newbool{solutions}
\newbool{grading}
\booltrue{solutions}
%\boolfalse{solutions}
\booltrue{grading}
%\boolfalse{grading}

\newcommand{\DoNotShare}{\large \noindent \textbf{Under the Harvey Mudd Honor Code, this document is not to be shared.} \normalsize}
\newcommand{\Problem}[3]{\mbox{} \newline \noindent \textbf{\textbf{Challenge #1: #2 [#3 Points] \\ }}}
\newrobustcmd{\Solution}[1]{\ifbool{solutions}{\mbox{} \newline \noindent \textbf{Solution:} #1}{}}
\newrobustcmd{\Grading}[1]{\ifbool{grading}{\mbox{} \newline \noindent \textbf{Grading Guidelines:} #1}{}}

\begin{document}

\begin{center}
	\bf
	Logic and Computability  \\
	Homework 6:  Distinguishability and Simulations \\
\end{center}

%\DoNotShare

\textbf{Please read Handout 2 before embarking on this assignment.} It shows the level of rigor expected for proofs on this assignment.

\section*{Challenges}

\Problem{1}{Proving that Languages are not Regular!}{20}

Use the \emph{Pairwise Distinguishability Theorem} to prove that each of the following languages is not regular.  Please read Handout 2 for an example of how to write such a proof.

\begin{enumerate}
	\item The language of all binary strings that contain exactly twice as many 0's as 1's.
	\item The language of binary strings $L = \{x | \ \exists y \in \{ 0, 1\}^{*} \ \mbox{s.t.}\ x = yy \}$.  Note that a string of the form $yy$ is a string $y$ appearing twice ``back-to-back'' as in $011011$ or $11$ or $10001000$.
\end{enumerate}

\Problem{2}{Multiples of Three in Binary}{25}

Use the Pairwise Distinguishability Theorem to prove that any DFA that accepts the language of multiples of three in binary requires at least three states. 

\Problem{3}{01 and 10!}{25}

Consider the language of binary strings in which the number of occurrences of \verb+01+ is equal to the number of occurrences of \verb+10+,  For example,
\verb+00011110+ is in the language because it has one \verb+01+ pattern and one \verb+10+ pattern.  The string \verb+01010+ is also in the language.  But, the string \verb+011001+ is not in the language.
Is this language regular?  If so, submit a Prolog file that represents the DFA (using the DFA representation that we used in HW \#5) called \verb+problem3.pl+.  If it's not a regular language, submit a proof using the Pairwise Distinguishability Theorem in a file called \verb+problem3.pdf+ using the same problem upload (we will grade your proof manually).
 
\Problem{4}{Variants of DFAs}{30}

Professor I.~Lai wakes up one morning and finds himself in a completely unfamiliar place.  He quickly realizes that he's been warped into another universe.  The one-eyed aliens there are very friendly, providing him with kiwi-lime-watermelon lollipops and showing him several of their intriguing models of computation.  For each of the following models of computation, indicate whether it's \textbf{more powerful} or has the \textbf{same power} as DFAs and prove your answer.  
	
\begin{enumerate}

	\item \textbf{Middle-Window DFAs:}  A Middle-Window DFA is like a DFA but the machine can always see the symbol at the middle of the tape.  If the length of the input string is even, there are two middle symbols, in which case the middle-window DFA sees the ``left middle''.  For example, if the input was ``milk'', the middle-window DFA would see the ``i''.  \emph{Note that each time the tape head moves to the right, the middle window floats to always be at the middle of the remaining string.  So, after reading the ``m'' in milk, the remaining string is ``ilk'' and the middle window now sees the ``l''.}  More formally, a Middle-Window DFA is a 5-tuple $(Q, \Sigma, \delta, q_0, F)$ where the components are analogous to a DFA but $\delta: Q \times \Sigma \times \Sigma \rightarrow Q$ is a transition function from the current state, current symbol, and middle window symbol to the next state.
	
	\item \textbf{Rear-Window DFAs:}  A Rear-Window DFA is like a DFA but the machine can always see the last symbol on the tape. A  Rear-Window DFA is a 
	5-tuple $(Q, \Sigma, \delta, q_0, F)$ where all the components are analogous to a DFA except that $\delta:  Q \times \Sigma \times \Sigma \rightarrow Q$ is a transition function from the current state, current symbol, and last symbol to the next state.
		
\end{enumerate}

\end{document}


